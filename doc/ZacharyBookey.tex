\documentclass{ccr15}

% PACKAGES ---------------------------------------------------------------
\usepackage{amsfonts,amsmath,graphicx,subfigure}
% ADD YOUR OWN PACKAGES HERE ---------------------------------------------
%\usepackage{someotherpackage}

% DEFINITIONS ------------------------------------------------------------
% ADD YOUR OWN DEFINITIONS HERE ------------------------------------------
% BE SURE TO PREFACE LABEL WITH YOUR OWN INITIALS (ZAB in this example) --
\newcommand{\ZABnorm}[1]{\left\Vert#1\right\Vert}
\newcommand{\ZABabs}[1]{\left\vert#1\right\vert}

% This controls the table-of-contents entry in the proceedings. Edit it
% to include your article title followed by the authors' names, as shown.
\addcontentsline{toc}{chapter}{The Quantum Mechanics of Chocolate Pudding\\
{\em Z.B.\ Student and I.D.\ Mentor and S.R \ Mentor}}

\pagestyle{myheadings}

\thispagestyle{plain}

% This gives the running head. Usually you list a shortened version of
% your article title (unless it's already very short) along with
% the author's names, as shown.
\markboth{Pudding Quantum Mechanics}{Z.B.\ Student and I.D.\ Mentor and S.R \ Mentor}

% Put your article title in here
\title{The Quantum Mechanics of Chocolate Pudding}

% List each author, their affiliation, and their e-mail address, as shown.
\author{Zachary A.\ Bookey\thanks{Saint John's University, zabookey@csbsju.edu} \and Irina P.\ Demeshko\thanks{Sandia National Laboratories,
ipdemes@sandia.gov} \and Sivasankaran Rajamanickam\thanks{Sandia National Laboratories, srajama@sandia.gov}}
\begin{document}

\maketitle

% Include your abstract here.
\begin{abstract}
The High Performance Conjugate Gradients (HPCG) Benchmark is an international project to create a
more accurate benchmark test for the world's largest computers. The current LINPACK benchmark,
which is the standard for measuring the performance of the top 500 fastest computers in the
world, is moving computers in a direction that is no longer beneficial to many important
parallel applications. In this project we are developing a version of HPCG, using the Kokkos
package found in Trilinos, that can be optimally executed across several distinct high
performance computing architectures. This new code demonstrates an efficient programming 
approach that can be adopted by other programmers to write portable high performance software.
\end{abstract}

\begin{itemize}
	\item Introduction
	\item HPCG
	\begin{itemize}
		\item History/Reason HPCG exists
		\item What HPCG does
		\begin{itemize}
			\item Methods and stuff
		\end{itemize}
	\end{itemize}
	\item Kokkos
	\begin{itemize}
		\item Maybe talk about the features of Kokkos
	\end{itemize}
	\item Explain the changes we made to HPCG to implement Kokkos
	\item Future Work
	\item Conclusion
\end{itemize}

\section{Introduction}
The High Performance Conjugate Gradient, from here out referred to as HPCG, is a new and
upcoming benchmark test to rank the worlds largest computers. HPCG uses a preconditioned 
conjugate gradient to solve a system of equations. The goal of our project was to create a
version of HPCG that uses the Kokkos package found in Trilinos in order to increase portability
among many types of architectures while maintaining reasonable performance.
\section{HPCG}
After generations of using the High Performance Linpack (HPL) benchmark to measure the
performance of large computers it became necessary to use another benchmark to help better the 
direction that super computers were headed to more accurately reflect the types of applications
that these machines were running. HPCG was created to fill the gap that HPL had created. On top
of solving a large system of equations, HPCG also features a more irregular data access pattern
so that data access affects results as well as matrix computations.

HPCG begins by creating a symmetric positive definite matrix and it's corresponding multi grid 
to be used in the preconditioning phase. For the preconditioner it uses a Symmetric Gauss-Seidel 
forward sweep and back sweep to solve the lower and upper triangular matrices. For the actual
solve of $A x = b$, HPCG uses the conjugate gradient method after the preconditioning phase.

I can add more info here if I need to.

\section{Kokkos}
As different computer architectures are better with certain applications than others it has
become increasingly difficult to write code that will perform well across many different types of
architectures. One solution to this problem is the C++ package, Kokkos. Kokkos acts as a wrapper
around your code to allow you to specify at compile time where and how you want to run your
application. Currently Kokkos supports the following execution spaces:
\begin{itemize}
	\item Serial
	\item PThreads
	\item OpenMP
	\item Cuda
\end{itemize}

Kokkos has two main features, views and parallel kernels. A view is essentially a wrapper around
an array of data that gives you the option to specify which execution space you want to store the
data on and allows you to choose what sort of memory access traits you wish this data to have.
Views also handle their own memory management via reference counting so that the view
automatically deallocates itself when all of the variables that reference it go out of scope,
thus making memory management much simpler across multiple devices.

There are three main parallel kernels: parallel_for, parallel_reduce, and parallel_scan. All of
these serve their own purpose and act as wrappers over how you would execute a section of code in
parallel over the respective exectuion space. For all of the parallel kernels you initiate the
kernel by passing in a functor that performs the desired parallel operation, as of C++11 lambdas
work as well and soon lamda functionality will be fully supported by cuda from host to device.

\subsection{Kokkos Features}

\section{HPCG + Kokkos}

\section{Future Work}

\section{Conclusion}

\section{Actual Content}

As we saw in Section 1, we all like chocolate pudding. This is where I wish
\textsf{$\setminus$jargonfill} worked. It would fill the page with meaningless technobabble so I could illustrate this
package. Instead, I'll talk about how to use quotations in latex. "Never use these quotations." ``Always use these,
instead.''

\section{Conclusions}
Herein, we repeat the abstract in past tense.

Unlike many other baked goods, chocolate pudding is subject to a myriad of interesting (and unique) effects on both the
meso and nano scales.  Understanding these phenomena is critical, not only to America's restaurant industry, but to
children everywhere.  We have examined these effects and have proposed new potential models which accurately capture
the material structure of chocolate pudding.

\bibliographystyle{siam}
% Edit the line below to be your first and last names.
\nocite{ZAB:Mentor05}
\bibliography{ZacharyBookey}

% Edit FirstnameLastname below to be your first and last names, but leave the line commented out.
% This line will help me merge bibliographies for the proceedings.
%\documentclass{ccr15}

% PACKAGES ---------------------------------------------------------------
\usepackage{amsfonts,amsmath,graphicx,subfigure}
% ADD YOUR OWN PACKAGES HERE ---------------------------------------------
%\usepackage{someotherpackage}

% DEFINITIONS ------------------------------------------------------------
% ADD YOUR OWN DEFINITIONS HERE ------------------------------------------
% BE SURE TO PREFACE LABEL WITH YOUR OWN INITIALS (ZAB in this example) --
\newcommand{\ZABnorm}[1]{\left\Vert#1\right\Vert}
\newcommand{\ZABabs}[1]{\left\vert#1\right\vert}

% This controls the table-of-contents entry in the proceedings. Edit it
% to include your article title followed by the authors' names, as shown.
\addcontentsline{toc}{chapter}{The Quantum Mechanics of Chocolate Pudding\\
{\em Z.B.\ Student and I.D.\ Mentor and S.R \ Mentor}}

\pagestyle{myheadings}

\thispagestyle{plain}

% This gives the running head. Usually you list a shortened version of
% your article title (unless it's already very short) along with
% the author's names, as shown.
\markboth{Pudding Quantum Mechanics}{Z.B.\ Student and I.D.\ Mentor and S.R \ Mentor}

% Put your article title in here
\title{The Quantum Mechanics of Chocolate Pudding}

% List each author, their affiliation, and their e-mail address, as shown.
\author{Zachary A.\ Bookey\thanks{Saint John's University, zabookey@csbsju.edu} \and Irina P.\ Demeshko\thanks{Sandia National Laboratories,
ipdemes@sandia.gov} \and Sivasankaran Rajamanickam\thanks{Sandia National Laboratories, srajama@sandia.gov}}

\begin{document}

\maketitle

% Include your abstract here.
\begin{abstract}
Unlike many other baked goods, chocolate pudding is subject to a myriad of interesting (and unique) effects on both the
meso and nano scales.  Understanding these phenomena is critical, not only to America's restaurant industry, but to
children everywhere.  We intend to examine these effects and propose new potential models which accurately capture the
material structure of chocolate pudding.
\end{abstract}

\section{Introduction} \label{ZAB:sec:intro}
I like chocolate pudding.  For that matter everyone else does too. This relationship is captured in \eqref{ZAB:eq:one},
\begin{equation}\label{ZAB:eq:one}
f(x) = x^{100-d},
\end{equation}
where $f$ is happiness in utils, $x$ is the quantity of chocolate pudding consumed in ounces and $d$ is the age of the
subject in years.

It has been argued that that chocolate pudding consumption by age is shown in Figure \ref{ZAB:fig:Mentor05}.  However,
we propose an alternative pudding consumption model, shown in Figure \ref{ZAB:fig:cos}.

\begin{figure}[hbh]
\begin{center}
\scalebox{0.5}{\includegraphics{plots/ZABsinx}} \caption{Chocolate pudding consumption model of
\cite{ZAB:Mentor05}}\label{ZAB:fig:Mentor05}
\end{center}\end{figure}

\begin{figure}[htb]
\begin{center}
\scalebox{0.5}{\includegraphics{plots/ZABcosx}} \caption{Proposed chocolate pudding consumption model}\label{ZAB:fig:cos}
\end{center}\end{figure}

We also want to show these figures side by side, to better illustrate our comparison.  We can see this in Figures
\ref{ZAB:fig:comp1} and \ref{ZAB:fig:comp2}.

\begin{figure}[htb]
\begin{center}
\subfigure[M. Mentor's Model]{\includegraphics[scale=.5]{plots/ZABsinx}\label{ZAB:fig:comp1}} \subfigure[Proposed
Model]{\includegraphics[scale=.5]{plots/ZABcosx}\label{ZAB:fig:comp2}} \caption{Comparative chocolate pudding consumption
models}
\end{center}\end{figure}

\section{Actual Content}

As we saw in Section \ref{ZAB:sec:intro}, we all like chocolate pudding. This is where I wish
\textsf{$\setminus$jargonfill} worked. It would fill the page with meaningless technobabble so I could illustrate this
package. Instead, I'll talk about how to use quotations in latex. "Never use these quotations." ``Always use these,
instead.''

\section{Conclusions}
Herein, we repeat the abstract in past tense.

Unlike many other baked goods, chocolate pudding is subject to a myriad of interesting (and unique) effects on both the
meso and nano scales.  Understanding these phenomena is critical, not only to America's restaurant industry, but to
children everywhere.  We have examined these effects and have proposed new potential models which accurately capture
the material structure of chocolate pudding.

\bibliographystyle{siam}
% Edit the line below to be your first and last names.
\bibliography{ZacharyBookey}

% Edit FirstnameLastname below to be your first and last names, but leave the line commented out.
% This line will help me merge bibliographies for the proceedings.
%\input{FirstnameLastname/FirstnameLastname.bbl}

\end{document}


\end{document}
